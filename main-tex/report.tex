\environment Report
\environment Booktabs
\environment TableParse
\environment Math
\environment Code
\environment RussianUnits
\environment PLC

\useURL[introCTX][https://github.com/contextgarden/not-so-short-introduction-to-context][][\tech{\hyphenatedurl{github.com/contextgarden/not-so-short-introduction-to-context}}]
\useURL[cld manual][https://www.pragma-ade.com/general/manuals/cld-mkiv.pdf][][\tech{\hyphenatedurl{www.pragma-ade.com/general/manuals/cld-mkiv.pdf}}]
\useURL[lua org][https://www.lua.org/about.html][][\tech{\hyphenatedurl{lua.org/about.html}}]
\useURL[peg][https://bford.info/pub/lang/peg.pdf][][\tech{\hyphenatedurl{bford.info/pub/lang/peg.pdf}}]
\useURL[lpeg][https://www.inf.puc-rio.br/~roberto/lpeg/][][\tech{\hyphenatedurl{www.inf.puc-rio.br/~roberto/lpeg/}}]
\useURL[bibliographies][https://www.pragma-ade.com/general/manuals/mkiv-publications.pdf][][\tech{\hyphenatedurl{www.pragma-ade.com/general/manuals/mkiv-publications.pdf}}]
\useURL[fonts context][https://www.pragma-ade.com/general/manuals/fonts-mkiv.pdf][][\tech{\hyphenatedurl{www.pragma-ade.com/general/manuals/fonts-mkiv.pdf}}]
\useURL[old context manual][https://www.pragma-ade.com/general/manuals/cont-eni.pdf][][\tech{\hyphenatedurl{www.pragma-ade.com/general/manuals/cont-eni.pdf}}]
\useURL[luametatex manual][https://www.pragma-ade.com/general/manuals/luametatex.pdf][][\tech{\hyphenatedurl{www.pragma-ade.com/general/manuals/luametatex.pdf}}]
\useURL[emuStudio official][https://www.emustudio.net/][][\tech{\hyphenatedurl{emustudio.net/}}]
\useURL[pretty assembler][https://github.com/svofski/pretty-8080-assembler][][\tech{\hyphenatedurl{github.com/svofski/pretty-8080-assembler}}]

\setupdocument[
    lab:topic={Тема},
    lab:author={Давыдов Кирилл Александрович},
]

%\usebtxdataset[report.bib]
%\nocite[*]

%\useURL[ms-sql-manual][link][][\tech{\hyphenatedurl{link}}]

%\setuppagenumbering[state=stop]

\startdocument

\startsectionblock[task]



\stopsectionblock

\startsectionblock[annotation]

\startexcludedtitle[
    title={Аннотация},
    reference={title: annotation},
]

В данной пояснительной записке приводится описание разработки ассемлбера и
генератора детализированных листингов для микропроцессора \corp{Intel 8080} в
виде интегрированного с текстовым процессором \corp{Con\abbr{\TeX}t} решения и
составленного на языке программирования \corp{Lua}. \abbr{ПО} предназначено для
облегчения разработки и документации программ, написанных для учебных
микропроцессорных стендов \corp{\abbr{УМПК}-80}. Работа также преследует
побочную цель демонстрации возможностей текстового процессора \corp{Con\abbr{\TeX}t}
как средства составления сложной документации, в том числе с применением
автоматизированной вёрстки.

Процесс разработки следует стандартной схеме и включает изучение языка
ассемлбера микропроцессоров \corp{Intel 8080}, реализацию парсера, ассемлбера и
инструментов форматирования для генерации листинга. Для изучения языка
ассемлбера применялось выпущенное в 1977 году официальное руководство. Для
составления парсера была применена библиотека \abbr{LPEG} для составления
парсеров на основе \abbr{РВ}-грамматики, включённая в доступные по умолчанию в
текстовом процессоре \corp{Con\abbr{\TeX}t} библиотеки. Демонстративные инструменты
форматирования используют интерфейс для взаимодействия с \corp{Con\abbr{\TeX}t} через
\corp{Lua}, но в будущем предполагается интерфейс, более приближенный к самому
текстовому процессору, и как следствие более удобный.

\corp{Con\abbr{\TeX}t} является весьма непопулярным текстовым процессором. Это можно
объяснить отсутствием маркетинга, необходимостью составления документов на
специальном (предметно-ориентированным) языке программирования, усиленным
фокусом на качество вёрстки ценой сложности инструмента. В связи с этим
возможности внедрения разработанного ассемлбера практически отсутствуют, пока
требования к качеству вёрстки в академической среде ограничены возможностями
более популярных текстовых процессоров, прежде всего \corp{Microsoft Office
Word}.

Полученное на момент написания пояснительной записки программное решение имеет
функционал, достаточный для использования его в указанных целях, но требует
доработки в полноценный модуль для текстового процессора для более удобной
работы. В частности, нужно реализовать поддержку ассемблерных директив,
составления программ отдельными сегментами, а также предоставить возможность
гибкой настройки вида листинга.

\stopexcludedtitle

\startexcludedtitle[
    title={Annotation},
    reference={excludedtitle: annotation en},
]

This explanatory note contains a description of the development process of an
assembler and code listing generator for the \corp{Intel 8080} line of
microprocessors in the form of a solution integrated with the \corp{Con\abbr{\TeX}t}
text processor and written in the \corp{Lua} programming language. This
software is intended to simplify the development and documentation of
programmes written for the \corp{\abbr{UMPK}-80} educational microprocessor
system. This work also intends to demonstrate the typesetting capabilities of
\corp{Con\abbr{\TeX}t} in the realm of documentation, including the usage of its
automated typesetting instruments.

The development process follows standard procedure and includes studying the
assembly language itself followed by developing an appropriate parser,
assembler and formatter capable of typesetting a code listing. The official
\corp{Intel 8080} assembly language manual is the main reference used. The
parser has been written using the \abbr{LPEG} library for writing parsers based
on parsing expression grammars. It comes by default with \corp{Con\abbr{\TeX}t}. A
demonstrative formatter has been developed that depends mainly on an \abbr{API}
that allows using \corp{Con\abbr{\TeX}t} functionality from \corp{Lua}, however the
final product will include a formatter that is more integrated with the text
processor itself.

\corp{Con\abbr{\TeX}t} is largely unpopular. It may be caused by a lack of marketing,
usage of a domain oriented language being a requirement for writing documents,
the complexity of the tool coming as a price for high quality typesetting. It
comes as no surprise that there is little hope for integrating the developed
software, not until academic typesetting requirements grow beyond the
possibilities of much more popular text processing software such as
\corp{Microsoft Office Word}.

At the moment of writing this explanatory note the developed software solution
is capable of achieving the given goals, however it requires further
development to make it easier to use. It is planned that the solution will
receive assembly directive support, literate programming style source code
splitting and more flexible code listing typesetting capabilities.

\stopexcludedtitle

\stopsectionblock

\startsectionblock[terms]

\startexcludedtitle[
    title={Обозначения и сокращения},
    reference={title: terms},
]

\setupxtable[booktabs][
    bodyfont=,
    columndistance=\spaceamount,
    setups=,
]

\startxtable[booktabs]

\startxtablebody

\startxrow

\startxcell[align=flushright]
\corp{Con\abbr{\TeX}t}
\stopxcell

\startxcell[align=center]
---
\stopxcell

\startxcell[align={width, bottom, fullhz, hanging}]
текстовый процессор на основе системы компьютерной вёрстки \corp{\abbr{\TeX}}.
[\in[item: introCTX]] Разрабатывается нидерландской компанией \abbr{PRAGMA
ADE}, занимающейся автоматизированной высококачественной вёрсткой. Использует
собственно разработанную новейшую версию \corp{\abbr{\TeX}} под названием
\corp{LuaMeta\abbr{\TeX}}, имеющую множество улучшений, в том числе встроенный
интерпретатор \corp{Lua}. [\in[item: cld manual]]
\stopxcell

\stopxrow

\startxrow

\startxcell[align=flushright]
\corp{Intel 8080}
\stopxcell

\startxcell[align=center]
---
\stopxcell

\startxcell[align={width, bottom, fullhz, hanging}]
второй восьмибитный микропроцессор от \corp{Intel}, выпуск которого начался в
апреле 1974 года.
\stopxcell

\stopxrow

\startxrow

\startxcell[align=flushright]
\abbr{LPEG}
\stopxcell

\startxcell[align=center]
---
\stopxcell

\startxcell[align={width, bottom, fullhz, hanging}]
\corp{Lua Parsing Expression Grammar}, библиотека для составления парсеров на
основе \abbr{РВ}-грамматики в \corp{Lua}. [\in[item: lpeg]]
\stopxcell

\stopxrow

\startxrow

\startxcell[align=flushright]
\corp{Lua}
\stopxcell

\startxcell[align=center]
---
\stopxcell

\startxcell[align={width, bottom, fullhz, hanging}]
легко встраиваемый и расширяемый интерпретируемый язык программирования.
Нередко используется в полноценных программных решениях как способ пользователю
расширить функционал программы. Отличается простотой, скоростью работы
интерпретатора и широким применением хеш-таблиц. [\in[item: lua org]]
\stopxcell

\stopxrow

\startxrow

\startxcell[align=flushright]
\abbr{РВ}-грамматика
\stopxcell

\startxcell[align=center]
---
\stopxcell

\startxcell[align={width, bottom, fullhz, hanging}]
грамматика, разбирающая выражение. Один из типов формальной грамматики.
Применяется для обработки компьютерных языков (см.\ также контекстно-свободную
грамматику). [\in[item: peg]]
\stopxcell

\stopxrow

\startxrow

\startxcell[align=flushright]
\abbr{УМПК}-80
\stopxcell

\startxcell[align=center]
---
\stopxcell

\startxcell[align={width, bottom, fullhz, hanging}]
учебно-методический стенд, предназначенный для практического изучения
микропроцессорных систем на начальном уровне. В основе лежит микропроцессор
\corp{Intel 8080}. [\in[item: umpk-80]] Используется в Сургутском
государственном университете преподавателями и студентами на направлениях
\quotation{Управление в технических системах} и \quotation{Программная
инженерия}.
\stopxcell

\stopxrow

\stopxtablebody

\stopxtable

\stopexcludedtitle

\stopsectionblock

\startsectionblock[contents]

\completecontent

\stopsectionblock

\startsectionblock[intro]

\starttitle[
    title={Введение},
    reference={title: intro},
]

Выполнение лабораторных работ с учебно-методическим стендом
\corp{\abbr{УМПК}-80} включает в себя анализ проблемы, проектирование решения,
последующую реализацию и тестирование с возможными доработки по мере
необходимости. Это проводится в целях закрепления учебного материала по
дисциплине \quotation{Организация \abbr{ЭВМ}}. [\in[item: umpk-80]] Далее
необходимо привести отчёт, одним из обязательных элементов которого является
детальный листинг в виде таблицы, включающей в себя адреса каждого байта
машинного кода программы, метки для наглядности, сам машинный код и
эквивалентный ассемлберный код, а также комментарии (\in{таблица}[table: code
listing example]).

\startplacetable[
    title={Пример листинга},
    reference={table: code listing example},
]

\startluacode
Intel8080.formatting.make_listing(
[==[
memcpy:
ldax d ;  Байт считывается из источника\ldots
mov m, a ; \ldots{} и он записывается по адресу назначения
inx d ; Следующая ячейка памяти источника\ldots
inx h ; Следующая ячейка памяти назначения\ldots
dcr c ; Одним байтом меньше, декремент счётчика
jnz memcpy ; Если есть ещё байты, продолжить копирование
ret ; Выход из подпрограммы
]==], 
0x0990
)
\stopluacode

\stopplacetable

Проблема составления листинга является актуальной, но второстепенной --- размер
программ, необходимых для выполнения лабораторных работ, достаточно мал, что
можно вручную выполнить ассемблирование. В связи с этим данная работа
выполняется больше как демонстрация возможностей текстового процессора
\corp{Con\abbr{\TeX}t}.

Несмотря на то, что формально листинг необходим лишь как элемент документации,
особенности ввода программ на учебном стенде [\in[item: umpk-80]] делают его
необходимым ещё на этапе реализации и даже реализации решения, прежде всего
позволяя сопоставить машинный код и адресное пространство системы для
последующего ввода программы в систему. Вариантов составления его не так много:

\startitemize

\startitem
Листинг можно составить вручную, что делает большинство студентов. Как правило,
для этого используются электронные таблицы (в частности \corp{Microsoft Office
Excel}). При аккуратном ведении листинга можно обеспечить правильную
последовательность используемых адресов и даже простоту исправления неверных
адресов и пропущенных команд, но процесс ассемблирования придётся
выполнять вручную. По очевидным причинам это несёт за собой вероятность
совершения ошибок.
\stopitem

\startitem
Можно воспользоваться уже существующими ассемблерами в дополнении к электронным
таблицам. В лучшем случае с помощью них можно получить ассемблерный код в виде
файла формата \corp{Intel \abbr{HEX}} (\in{рисунок}[figure: intel hex]). Как
можно увидеть, он является достаточно неудобным для копирования, но
действительно упрощает процесс ассемблирования. В частности, можно спутать
контрольную сумму с байтом машинного кода программы; приведённые в файле
(двухбайтовые) адреса представлены в порядке от старшего байта к младшему в
отличие от используемого в микропроцессоре обратного порядка. Сгенерировать
файлы этого формата возможно, например, с помощью инструментов эмулятора
\corp{emuStudio} [\in[item: emuStudio]]. Есть и онлайн-инструменты, как
\corp{Pretty Intel 8080 Assembler} [\in[item: pretty assembler]].

\startplacefigure[
    title={Формат \corp{Intel \abbr{HEX}}. Выделены по порядку: количество байт, адрес, тип записи, данные, контрольная сумма},
    reference={figure: intel hex},
    minwidth=\textwidth,
]

\externalfigure[intel-hex-map.png][width=.7\textwidth]

\stopplacefigure
\stopitem

\startitem
В редком случае можно найти инструменты, генерирующие листинг в неком виде. В
частности, в работе \angled{name} [\in[item: umpk emu]] представлен эмулятор,
имеющий также функционал для генерации листингов в формате, приведённом в
\in{таблице}[table: code listing example]. Это значительно облегчает
составление отчёта, но требуют повторной генерации вне текстовых процессоров в
случае изменения программы.
\stopitem

\stopitemize

Необходимо отметить, что решения с применением ассемлбера предполагают наличие
отдельного от текстового процессора \abbr{ПО}. В данной работе представляется
решение проблемы, внедрённое в текстовый процессор и позволяющее генерировать
листинги прямиком в документе.

\stoptitle

\stopsectionblock

\startsectionblock[main]

\starttitle[
    title={Основная часть},
    reference={title: main},
]



\stoptitle

\stopsectionblock

\startsectionblock[conclude]

\starttitle[
    title={Заключение},
    reference={title: conclude},
]



\stoptitle

\stopsectionblock

\startsectionblock[sources]

\starttitle[
    title={Список использованных источников},
    reference={title: sources},
]

\startitemize[n]

\startitem[item: introCTX]
Атаз-Лопес Х. A not so short introduction to \corp{Con\abbr{\TeX}t{} Mark IV} [Электронный ресурс] : неофициальный справочник по \corp{Con\abbr{\TeX}t} для начинающих / [перевод с испанского анонимным лицом]. --- [Б.\,м.], 2021. --- 307 с. --- URL: \from[introCTX] (дата обращения 08.06.2025)
\stopitem

\startitem[item: cld manual]
Хаген Г. \corp{Con\abbr{\TeX}t{}} \corp{Lua} documents [Электронный ресурс] : справочная информация по системе вёрстки \corp{LuaMeta\abbr{\TeX}} --- Хасселт: \abbr{PRAGMA ADE}, 2023. --- 214 с. --- URL: \from[cld manual] (дата обращения 08.06.2025)
\stopitem

\startitem[item: lpeg]
\abbr{LPEG} - Parsing Expression Grammars For Lua [Электронный ресурс] : главная страница проекта \abbr{LPEG} для создания парсеров. --- URL: \from[lpeg] (дата обращения 08.06.2025)
\stopitem

\startitem[item: lua org]
\corp{Lua}: about [Электронный ресурс] : базовые сведения о языке программирования \corp{Lua}. --- URL: \from[lua org] (дата обращения 08.06.2025)
\stopitem

\startitem[item: peg]
Форд Б. Parsing Expression Grammars: A Recognition-Based Syntactic Foundation [Электронный ресурс] : описание \abbr{РВ}-грамматики. // \abbr{POPL}: Венеция, 2004. --- 12 с. --- URL: \from[peg] (дата обращения 08.06.2025)
\stopitem

\startitem[item: umpk-80]
Запевалов А.\,В., Запевалова Л.\,Ю., Степанова Д.\,П. Методические указания к выполнению лабораторных работ по дисциплинам: \quotation{Вычислительные машины, системы и сети}, \quotation{Организация \abbr{ЭВМ} и систем} // Сургутский государственный университет \abbr{ХМАО}--Югры. --- Сургут, \abbr{ИЦ} Сур\abbr{ГУ}, 2020. --- 52 с.
\stopitem

\startitem[item: i8080]
\corp{Intel 8080/8085} Assembly Language Programming : документация по языку ассемблера для микропроцессоров \corp{Intel} 8080 и 8085. --- \corp{Intel Corporation}, 1978. --- 224 с.
\stopitem

\startitem[item: emuStudio]
\from[emuStudio official]
\stopitem

\startitem[item: pretty assembler]
\from[pretty assembler]
\stopitem

\startitem[item: bibliographies]
Хаген Г. Bibliographies the \corp{Con\abbr{\TeX}t} Way [Электронный ресурс] : справочная информация по библиографическому модулю \corp{Con\abbr{\TeX}t} --- Хасселт: \abbr{PRAGMA ADE}, 2017. --- 102 с. --- URL: \from[bibliographies] (дата обращения 08.06.2025)
\stopitem

\startitem[item: fonts context]
Хаген Г. Fonts Out of \corp{Con\abbr{\TeX}t} [Электронный ресурс] : справочная информация по работе со шрифтами в \corp{Con\abbr{\TeX}t} --- Хасселт: \abbr{PRAGMA ADE}, 2016. --- 228 с. --- URL: \from[fonts context] (дата обращения 08.06.2025)
\stopitem

%\startitem[item: old context manual]
%Хаген Г. \corp{Con\abbr{\TeX}t}: The Manual [Электронный ресурс] : официальный справочник по \corp{Con\abbr{\TeX}t} --- Хасселт: \abbr{PRAGMA ADE}, 2001. --- 369 с. --- URL: \from[old context manual] (дата обращения 08.06.2025)
%\stopitem
%
%\startitem[item: luametatex manual]
%Хаген Г. \corp{LuaMeta\abbr{\TeX}}: The Manual [Электронный ресурс] : документация по \corp{LuaMeta\abbr{\TeX}} --- Хасселт: \abbr{PRAGMA ADE}, 2025. --- 646 с. --- URL: \from[luametatex manual] (дата обращения 08.06.2025)
%\stopitem

\stopitemize

\stoptitle

\stopsectionblock

\startappendices

\startappendixchapter[
    title={Характеристика текстового процессора \corp{Con\abbr{\TeX}t}},
    bookmark={Характеристика текстового процессора ConTeXt},
    reference={appendixchapter: context desc},
]

Высококачественная типография имеет значительную важность для специалистов
технических направлений. Одним из актуальных выборов \abbr{ПО} для качественной
вёрстки является пакет программ \corp{Con\abbr{\TeX}t}.

\corp{Con\abbr{\TeX}t}---продвинутый текстовый процессор, разрабатываемый компанией
\corp{\abbr{PRAGMA} Advanced Document Engineering}, или \abbr{PRAGMA ADE}.
[\in[item: introCTX]] Построенный на новейшей версии системы компьютерной
вёрстки \corp{\abbr{\TeX}} под названием \corp{LuaMeta\abbr{\TeX}} или \abbr{LMTX},
\corp{Con\abbr{\TeX}t} обеспечивает вёрстку проектов-документов высокой сложности
благодаря следующим принципам и функционалу:

\startitemize[n]

\startitem

Автоматическое формирование библиографий исходя из содержимого выбранной базы
источников (файл) [\in[item: bibliographies]];

\stopitem

\startitem

Принцип \abbr{WYSIWYM} (\quotation{видишь то, что имеешь в виду}) вместо
\abbr{WYSIWYG} (\quotation{видишь то, что получишь}) при составлении
документов. Это значит, что пользователь работает не над готовым документом,
как в визуальных редакторах (пр.\ \corp{Microsoft Word}), а над входным файлом
с командами для \corp{Con\abbr{\TeX}t} и текстом на вывод в документ.
[\in[item: introCTX]] Этот файл обрабатывается \corp{Con\abbr{\TeX}t} и на
выход получается файл \abbr{PDF};

\stopitem

\startitem

Семантический подход к элементам документа. Это значит, что, например,
раздел---не просто текст, начинающийся с заголовка с определённым стилем шрифта
и выравниванием, а логический элемент входного файла, границы которого
определяются командами начала и конца раздела. Это делает входной файл более
понятным---не нужно знать, как выглядит заголовок раздела, чтобы понять, что
раздел начинается с \inlinecontextcode{\startsection}, заканчивается
\inlinecontextcode{\stopsection}, а переданный первой команде параметр
\inlinecontextcode{[title={Текст заголовка}]} задаёт текст заголовка;
[\in[item: introCTX]]

\stopitem

\startitem

Значительная пластичность и очевидная закономерность логических элементов. Для
их большинства, имея некий базовый элемент \tech{element}, будут команды
создания нового подвида элемента \inlinecontextcode{\defineelement}, настройки
элемента в целом или подвида на выбор \inlinecontextcode{\setupelement}, а
также обеспечивается использование элемента с помощью одной команды
\inlinecontextcode{\element} и/или с помощью команд начала и конца
\inlinecontextcode{\startelement} и \inlinecontextcode{\stopelement}
соответственно. Это позволяет легко изменить стиль целой группе элементов
текста одновременно; [\in[item: introCTX]]

\stopitem

\startitem

Обширная по возможностям настройка шрифта. По умолчанию \corp{Con\abbr{\TeX}t}
следит за четырьмя шрифтами на документ: антиквой (с засечками, основной шрифт
для текста), гротеском (без засечек), моноширинным (для кода) и математическим
(для формул). Наборы шрифтов---как этих четырёх, так и
дополнительных---называются машинописями. Пользователь способен объявлять
собственные машинописи, но есть и заранее готовые, идущие вместе с программой.
[\in[item: fonts context]] Позволяется включать любые доступные шрифту
особенности (англ.\ \emph{font features}) в пределах используемого формата
шрифта. Так, например, у \corp{Times New Roman} есть как маюскульные цифры
{\feature[+][lpnum]0123456789}, так и минускульные цифры
{\feature[+][opnum]0123456789}. Есть и дроби \fffrac{0123456789}{0123456789}.
Также поддерживается микротипографика---улучшение читаемости текста за счёт
висячих символов и небольшого сжатия/растягивания глифов. [\in[item: old
context manual]]

\stopitem

\startitem

Продвинутая математическая вёрстка. \corp{Con\abbr{\TeX}t} использует всю мощь
\corp{\abbr{\TeX}} для формирования математических конструкций со сложными
символами, растущими разграничителями и тому подобное (\in{рисунок}[figure:
math example]);

\startplacefigure[
    title={Пример математической вёрстки},
    reference={figure: math example},
]

\startformula

%\showboxes

U_2 = \frac{1}{2!}
\int_0^\beta \dd\tau_1 \int_0^\beta \dd\tau_2\;
\sum_{\startsubstack k_1,q_1 \NR k_2,q_2 \stopsubstack}
\Bigl\langle
\startmathalignment[align={1:right, 2:left},location={top, packed},align=left]
\NC
{{\mathcal T}} \NC \Bigl[
c_{k_1}^\dagger (\tau_1)
\Delta_{k_1,q_1}^r c_{-k_1}^* (\tau_1) + c_{-q_1}^T (\tau_1)
\Delta_{k_1,q_1}^{r\dagger} c_{q_1} (\tau_1)
\Bigr]
\NR
\NC
\times\; \NC \Bigl[
c_{k_2}^\dagger(\tau_2) \Delta_{k_2,q_2}^r c_{-k_2}^*
(\tau_2) + c_{-q_2}^T (\tau_2) \Delta_{k_2,q_2}^{r\dagger}
c_{q_2} (\tau_2)
\Bigr] \Bigr\rangle.
\NR
\stopmathalignment

\stopformula

\stopplacefigure

\stopitem

\startitem

Автоматическая нумерация разделов, подразделов, рисунков, таблиц и так далее;

\stopitem

\startitem

Использование языка программирования \corp{Lua} для расширения функционала

\stopitem

\stopitemize

\noindentation и так далее.

Будучи многофункциональным текстовым процессором с возможностью расширения его
способностей с помощью легкодоступного языка, решение задачи сборки программы
на неком языке в целях представления её в удобочитаемом виде не представляется
сложным.

\stopappendixchapter

\stopappendices

\stopdocument
